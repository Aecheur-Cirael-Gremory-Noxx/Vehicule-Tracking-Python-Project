\documentclass[12pt,a4paper]{article}

% Packages
\usepackage[utf8]{inputenc}
\usepackage[french]{babel}
\usepackage[T1]{fontenc}
\usepackage{geometry}
\usepackage{graphicx}
\usepackage{fancyhdr}
\usepackage{titlesec}
\usepackage{tocloft}
\usepackage{lipsum}
\usepackage{hyperref}
\usepackage{xcolor}
\usepackage{listings}
\usepackage{array}
\usepackage{tabularx}

% Configuration de la page
\geometry{left=3cm,right=2cm,top=3cm,bottom=3cm}

% Configuration des en-têtes et pieds de page
\pagestyle{fancy}
\fancyhf{}
\fancyhead[L]{\leftmark}
\fancyhead[R]{\thepage}
\fancyfoot[C]{\textit{Cahier des charges - Projet de suivi de véhicules}}

% Configuration des liens
\hypersetup{
    colorlinks=true,
    linkcolor=blue,
    filecolor=magenta,      
    urlcolor=cyan,
    citecolor=green
}

% Configuration du listing de code
\lstset{
    basicstyle=\ttfamily\footnotesize,
    breaklines=true,
    frame=single,
    language=Python,
    showstringspaces=false,
    commentstyle=\color{green},
    keywordstyle=\color{blue},
    stringstyle=\color{red}
}

% Titre du document
\title{
    \vspace{-2cm}
    \Huge\textbf{CAHIER DES CHARGES}\\
    \vspace{0.5cm}
    \Large Système de Suivi de Véhicules\\
    \vspace{0.3cm}
    \large Projet Python
}

\author{
    \textbf{Nom de l'étudiant:} [À compléter]\\
    \textbf{Formation:} [À compléter]\\
    \textbf{Établissement:} [À compléter]\\
    \textbf{Année académique:} 2024-2025
}

\date{\today}

\begin{document}

% Page de titre
\maketitle
\thispagestyle{empty}

\vfill

\begin{center}
    \includegraphics[width=0.3\textwidth]{logo_etablissement.png} % À remplacer par le logo de votre établissement
\end{center}

\vfill

\newpage

% Table des matières
\tableofcontents
\newpage

% 1. Introduction
\section{Introduction}

\subsection{Contexte du projet}

Dans le cadre d'un projet académique de 20 semaines proposé par l'Académie Militaire Technique de Bucarest, nous développons une application de surveillance routière intelligente. Les statistiques mondiales révèlent que les comportements de conduite agressifs contribuent à plus de 56\% des accidents mortels, avec 1,35 million de décès annuels selon l'Organisation Mondiale de la Santé.

La surveillance du trafic routier représente un enjeu crucial pour :

\begin{itemize}
    \item \textbf{La sécurité publique} : Prévention des accidents et protection des usagers
    \item \textbf{L'application de la loi} : Détection automatique des infractions routières
    \item \textbf{La gestion du trafic} : Optimisation des flux et identification des zones à risque
    \item \textbf{La collecte de données} : Analyse comportementale pour l'amélioration des infrastructures
\end{itemize}

Ce projet s'inscrit dans une démarche d'apprentissage pratique des technologies de vision par ordinateur et d'intelligence artificielle appliquées à des problématiques réelles de sécurité routière.

\subsection{Objectifs}

\subsubsection{Objectif principal}
Concevoir et développer une application logicielle capable de détecter et suivre les véhicules à partir d'un flux vidéo provenant d'une caméra de surveillance routière fixe, en identifiant les comportements dangereux et les infractions.

\subsubsection{Objectifs spécifiques}

\begin{itemize}
    \item \textbf{Détection et suivi} : Identifier et suivre en temps réel tous les véhicules dans le champ de vision
    \item \textbf{Reconnaissance} : Extraire les caractéristiques des véhicules (forme, couleur, plaque d'immatriculation)
    \item \textbf{Analyse comportementale} : Détecter les trajectoires suspectes et comportements agressifs
    \item \textbf{Identification d'infractions} : Reconnaître les violations des règles de circulation
    \item \textbf{Alertes} : Signaler automatiquement les anomalies détectées
\end{itemize}

\subsection{Portée du projet}

\subsubsection{Périmètre fonctionnel}
L'application couvrira :

\begin{itemize}
    \item \textbf{Détection multi-véhicules} : Traitement simultané de plusieurs véhicules
    \item \textbf{Suivi de trajectoire} : Analyse du mouvement dans le temps
    \item \textbf{Classification comportementale} :
    \begin{itemize}
        \item Conduite normale
        \item Comportement agressif (accélérations/freinages brusques, zigzag)
        \item Trajectoires suspectes
    \end{itemize}
    \item \textbf{Détection d'infractions} :
    \begin{itemize}
        \item Excès de vitesse
        \item Non-respect des distances de sécurité
        \item Changements de voie dangereux
        \item Franchissement de ligne continue
    \end{itemize}
\end{itemize}

\subsubsection{Limites du projet}

\textbf{Contraintes techniques :}
\begin{itemize}
    \item Caméra fixe uniquement (pas de suivi mobile)
    \item Traitement en différé acceptable (pas d'exigence temps réel strict)
    \item Conditions météorologiques normales
\end{itemize}

\textbf{Contraintes académiques :}
\begin{itemize}
    \item Durée limitée à 20 semaines
    \item Ressources étudiantes
    \item Focus sur la démonstration de faisabilité
\end{itemize}

% 2. Analyse de l'existant
\section{Analyse de l'existant}

\subsection{Solutions concurrentes}

\subsubsection{Solutions commerciales}

\textbf{Systèmes de contrôle automatisé} : Radars fixes avec reconnaissance de plaques
\begin{itemize}
    \item \textit{Avantages} : Précision, fiabilité prouvée
    \item \textit{Limites} : Coût élevé, détection limitée aux excès de vitesse
\end{itemize}

\textbf{Caméras intelligentes} : Solutions intégrées type "smart cameras"
\begin{itemize}
    \item \textit{Avantages} : Traitement embarqué, détection multi-infractions
    \item \textit{Limites} : Prix prohibitif pour un projet étudiant
\end{itemize}

\subsubsection{Projets de recherche}

\textbf{SHRP2 Naturalistic Driving Study} : Base de données comportementales
\begin{itemize}
    \item \textit{Pertinence} : Référence pour la classification des comportements
    \item \textit{Limitation} : Données embarquées, non adaptées aux caméras fixes
\end{itemize}

\textbf{D³ System} : Détection fine de comportements anormaux
\begin{itemize}
    \item \textit{Intérêt} : Taxonomie des comportements (6 types)
    \item \textit{Complexité} : Trop avancé pour notre délai
\end{itemize}

\subsection{Technologies disponibles}

\subsubsection{Détection et suivi d'objets}

\textbf{YOLO (You Only Look Once)} : Détection temps réel
\begin{itemize}
    \item \textit{Avantages} : Rapide, précis, bien documenté
    \item \textit{Adaptation} : Versions légères (YOLOv5, YOLOv8) adaptées au projet
\end{itemize}

\textbf{Méthodes traditionnelles} : Filtres de Kalman pour le suivi
\begin{itemize}
    \item \textit{Avantages} : Léger, éprouvé, simple à implémenter
    \item \textit{Usage} : Complémentaire à YOLO pour le suivi temporel
\end{itemize}

\subsubsection{Analyse comportementale}

\textbf{Approches statistiques} :
\begin{itemize}
    \item Seuillage sur accélérations/décélérations
    \item Analyse de trajectoire par régression
    \item \textit{Avantage} : Simple, interprétable
\end{itemize}

\textbf{Machine Learning basique} :
\begin{itemize}
    \item SVM pour classification binaire (normal/agressif)
    \item Random Forest pour détection multi-classes
    \item \textit{Compromis} : Performance acceptable avec complexité maîtrisée
\end{itemize}

\subsubsection{Reconnaissance de plaques}

\textbf{OpenALPR} : Bibliothèque open-source
\begin{itemize}
    \item \textit{Avantages} : Gratuit, performant, multi-régions
    \item \textit{Intégration} : API simple pour notre application
\end{itemize}

\subsubsection{Frameworks de développement}

\begin{itemize}
    \item \textbf{OpenCV} : Vision par ordinateur
    \item \textbf{Python} : Prototypage rapide avec écosystème riche
    \item \textbf{TensorFlow/PyTorch} : Si apprentissage profond nécessaire
\end{itemize}

Cette approche pragmatique permet de construire une solution fonctionnelle dans le délai imparti, tout en offrant des possibilités d'évolution future.

% 3. Spécifications fonctionnelles
\section{Spécifications fonctionnelles}

\subsection{Besoins utilisateurs}

\subsubsection{Utilisateur administrateur}
\begin{itemize}
    \item Gestion des véhicules (ajout, suppression, modification)
    \item Configuration du système
    \item Consultation des rapports
    \item Gestion des utilisateurs
\end{itemize}

\subsubsection{Utilisateur standard}
\begin{itemize}
    \item Visualisation de la position des véhicules
    \item Consultation de l'historique
    \item Génération de rapports simples
\end{itemize}

\subsection{Fonctionnalités principales}

\subsubsection{Suivi en temps réel}
\lipsum[6]

\subsubsection{Historique des déplacements}
\lipsum[7]

\subsubsection{Alertes et notifications}
\lipsum[8]

\subsection{Cas d'utilisation}

\begin{figure}[h]
\centering
% Ici, vous pourriez insérer un diagramme de cas d'utilisation
\framebox[0.8\textwidth][c]{Diagramme de cas d'utilisation à insérer}
\caption{Diagramme de cas d'utilisation principal}
\end{figure}

% 4. Spécifications techniques
\section{Spécifications techniques}

\subsection{Architecture du système}

\lipsum[9]

\subsubsection{Architecture globale}
Le système sera développé selon une architecture modulaire comprenant :
\begin{itemize}
    \item Module de collecte des données GPS
    \item Module de traitement et d'analyse
    \item Module de stockage (base de données)
    \item Module d'interface utilisateur
    \item Module de communication
\end{itemize}

\subsection{Technologies utilisées}

\subsubsection{Langage de programmation}
\textbf{Python 3.x} - Choisi pour sa simplicité et ses nombreuses bibliothèques.

\subsubsection{Frameworks et bibliothèques}
\begin{itemize}
    \item \textbf{Flask/Django} : Framework web
    \item \textbf{SQLAlchemy} : ORM pour la base de données
    \item \textbf{Folium/Leaflet} : Cartographie
    \item \textbf{Pandas} : Manipulation de données
    \item \textbf{NumPy} : Calculs numériques
\end{itemize}

\subsection{Base de données}

\lipsum[10]

\begin{lstlisting}[caption=Exemple de structure de table véhicule]
CREATE TABLE vehicules (
    id INTEGER PRIMARY KEY,
    immatriculation VARCHAR(20) NOT NULL,
    marque VARCHAR(50),
    modele VARCHAR(50),
    date_creation TIMESTAMP DEFAULT CURRENT_TIMESTAMP
);
\end{lstlisting}

% 5. Contraintes
\section{Contraintes}

\subsection{Contraintes techniques}

\begin{itemize}
    \item Compatibilité avec les navigateurs modernes
    \item Performance : temps de réponse < 2 secondes
    \item Disponibilité : 99.5\% minimum
    \item Sécurité des données personnelles (RGPD)
\end{itemize}

\subsection{Contraintes temporelles}

\lipsum[11]

\begin{table}[h]
\centering
\begin{tabular}{|l|l|l|}
\hline
\textbf{Phase} & \textbf{Durée} & \textbf{Livrable} \\
\hline
Analyse & 2 semaines & Cahier des charges \\
\hline
Conception & 3 semaines & Spécifications techniques \\
\hline
Développement & 8 semaines & Application fonctionnelle \\
\hline
Tests & 2 semaines & Rapport de tests \\
\hline
Déploiement & 1 semaine & Système en production \\
\hline
\end{tabular}
\caption{Planning prévisionnel}
\end{table}

\subsection{Contraintes budgétaires}

\lipsum[12]

% 6. Interface utilisateur
\section{Interface utilisateur}

\subsection{Maquettes}

\lipsum[13]

\begin{figure}[h]
\centering
\framebox[0.8\textwidth][c]{Maquette de l'interface principale à insérer}
\caption{Interface principale de l'application}
\end{figure}

\subsection{Ergonomie}

\lipsum[14]

% 7. Tests et validation
\section{Tests et validation}

\subsection{Stratégie de tests}

\subsubsection{Tests unitaires}
\lipsum[15]

\subsubsection{Tests d'intégration}
\lipsum[16]

\subsubsection{Tests fonctionnels}
\lipsum[17]

\subsection{Critères d'acceptation}

\begin{itemize}
    \item Toutes les fonctionnalités principales opérationnelles
    \item Interface utilisateur conforme aux maquettes
    \item Performance respectée
    \item Sécurité validée
\end{itemize}

% 8. Livrables
\section{Livrables}

\subsection{Documentation}

\begin{itemize}
    \item Cahier des charges (ce document)
    \item Documentation technique
    \item Manuel utilisateur
    \item Guide d'installation
\end{itemize}

\subsection{Code source}

\lipsum[18]

\subsection{Tests}

\lipsum[19]

% 9. Maintenance et évolution
\section{Maintenance et évolution}

\subsection{Maintenance corrective}

\lipsum[20]

\subsection{Évolutions prévues}

\begin{itemize}
    \item Intégration de nouvelles sources de données
    \item Amélioration de l'interface mobile
    \item Ajout de fonctionnalités d'analyse prédictive
    \item Extension à d'autres types de véhicules
\end{itemize}

% 10. Conclusion
\section{Conclusion}

\lipsum[21-22]

Ce cahier des charges définit les bases solides pour le développement d'un système de suivi de véhicules moderne et efficace. Le respect de ces spécifications garantira la livraison d'une solution répondant aux besoins identifiés.

% Annexes
\newpage
\appendix

\section{Glossaire}

\begin{description}
    \item[API] Application Programming Interface
    \item[GPS] Global Positioning System
    \item[ORM] Object-Relational Mapping
    \item[RGPD] Règlement Général sur la Protection des Données
    \item[UI/UX] User Interface / User Experience
\end{description}

\section{Références}

\begin{enumerate}
    \item Documentation Python : \url{https://docs.python.org/}
    \item Flask Framework : \url{https://flask.palletsprojects.com/}
    \item Folium Documentation : \url{https://folium.readthedocs.io/}
    \item Règlement RGPD : \url{https://gdpr.eu/}
\end{enumerate}

\end{document}